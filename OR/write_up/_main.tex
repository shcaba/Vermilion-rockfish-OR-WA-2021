\PassOptionsToPackage{enable-debug,check-declarations}{expl3}
\RequirePackage{pdfmanagement-testphase}
\DeclareDocumentMetadata {  }
\ExplSyntaxOn
\pdfmanagement_add:nnn{Catalog}{Lang}{(enUS)}
\ExplSyntaxOff

% xmp metadata for pdf
% Originally used \usepackage[a-2a]{pdfx}
% \usepackage{hyperxmp} replaced it
% \RequirePackage{pdfmanagement-testphase} replaced it

\documentclass[11pt,
  english,
  a4paper,
]{article}
\usepackage{sa4ss}
\usepackage{amsmath,amssymb,array}
\usepackage{booktabs}

% From tagged-template.latex
\usepackage{lmodern}
\usepackage{ifxetex,ifluatex}
\ifnum 0\ifxetex 1\fi\ifluatex 1\fi=0 % if pdftex
  \usepackage[T1]{fontenc}
  \usepackage[utf8]{inputenc}
  \usepackage{textcomp} % provide euro and other symbols
\else % if luatex or xetex
  \usepackage{unicode-math}
  \defaultfontfeatures{Scale=MatchLowercase}
  \defaultfontfeatures[\rmfamily]{Ligatures=TeX,Scale=1}
\fi

% Use upquote if available, for straight quotes in verbatim environments
\IfFileExists{upquote.sty}{\usepackage{upquote}}{}
\IfFileExists{microtype.sty}{% use microtype if available
  \usepackage[]{microtype}
  \UseMicrotypeSet[protrusion]{basicmath} % disable protrusion for tt fonts
}{}
\makeatletter
\@ifundefined{KOMAClassName}{% if non-KOMA class
  \IfFileExists{parskip.sty}{%
    \usepackage{parskip}
  }{% else
    \setlength{\parindent}{0pt}
    \setlength{\parskip}{6pt plus 2pt minus 1pt}}
}{% if KOMA class
  \KOMAoptions{parskip=half}}
\makeatother
\usepackage{xcolor}
\IfFileExists{xurl.sty}{\usepackage{xurl}}{} % add URL line breaks if available
\hypersetup{
  pdftitle={Status of Vermilion rockfish (Sebastes miniatus) along the US West - Oregon coast in 2021},
  pdflang={en},
  hidelinks,
  pdfcreator={LaTeX via pandoc}}
\urlstyle{same} % disable monospaced font for URLs
\usepackage{longtable}
% Correct order of tables after \paragraph or \subparagraph
\usepackage{etoolbox}
\makeatletter
\patchcmd\longtable{\par}{\if@noskipsec\mbox{}\fi\par}{}{}
\makeatother
% Allow footnotes in longtable head/foot
\IfFileExists{footnotehyper.sty}{\usepackage{footnotehyper}}{\usepackage{footnote}}
\makesavenoteenv{longtable}
\usepackage{graphicx}
\makeatletter
\def\maxwidth{\ifdim\Gin@nat@width>\linewidth\linewidth\else\Gin@nat@width\fi}
\def\maxheight{\ifdim\Gin@nat@height>\textheight\textheight\else\Gin@nat@height\fi}
\makeatother
% Scale images if necessary, so that they will not overflow the page
% margins by default, and it is still possible to overwrite the defaults
% using explicit options in \includegraphics[width, height, ...]{}
\setkeys{Gin}{width=\maxwidth,height=\maxheight,keepaspectratio}
% Set default figure placement to htbp
\makeatletter
\def\fps@figure{htbp}
\makeatother
\setlength{\emergencystretch}{3em} % prevent overfull lines
\providecommand{\tightlist}{%
  \setlength{\itemsep}{0pt}\setlength{\parskip}{0pt}}
\setcounter{secnumdepth}{5}
\ifxetex
  % Load polyglossia as late as possible: uses bidi with RTL langages (e.g. Hebrew, Arabic)
  \usepackage{polyglossia}
  \setmainlanguage[]{english}
\else
  \usepackage[shorthands=off,main=english]{babel}
\fi

%Define cslreferences environment, required by pandoc 2.8
%https://github.com/rstudio/rmarkdown/issues/1649
\newlength{\csllabelwidth}
\setlength{\csllabelwidth}{3em}
\newlength{\cslhangindent}
\setlength{\cslhangindent}{1.5em}
% for Pandoc 2.8 to 2.10.1
\newenvironment{cslreferences}%
  {}%
  {\par}
% For Pandoc 2.11+
\newenvironment{CSLReferences}[2] % #1 hanging-ident, #2 entry spacing
 {% don't indent paragraphs
  \setlength{\parindent}{0pt}
  % turn on hanging indent if param 1 is 1
  \ifodd #1 \everypar{\setlength{\hangindent}{\cslhangindent}}\ignorespaces\fi
  % set entry spacing
  \ifnum #2 > 0
  \setlength{\parskip}{#2\baselineskip}
  \fi
 }%
 {}
\usepackage{calc}  % for \widthof, \maxof in minipage
\newcommand{\CSLBlock}[1]{#1\hfill\break}
\newcommand{\CSLLeftMargin}[1]{\parbox[t]{\csllabelwidth}{#1}}
\newcommand{\CSLRightInline}[1]{\parbox[t]{\linewidth - \csllabelwidth}{#1}\break}
\newcommand{\CSLIndent}[1]{\hspace{\cslhangindent}#1}


\providecommand{\tightlist}{%
  \setlength{\itemsep}{0pt}\setlength{\parskip}{0pt}}


\date{}
\newcommand{\trTitle}{Status of Vermilion rockfish (\emph{Sebastes miniatus}) along the US West - Oregon coast in 2021}
\newcommand{\trYear}{2021}
\newcommand{\trMonth}{June}
\newcommand{\trAuthsLong}{truetrue}
\newcommand{\trAuthsBack}{Cope, J.M., A.D. Whitman}
\newcommand{\trCitation}{
\begin{hangparas}{1em}{1}
\trAuthsBack{}. \trYear{}. \trTitle{}. Pacific Fisheries Management Council, Portland, Oregon. \pageref{LastPage}{}\,p.
\end{hangparas}}

\AtBeginDocument{\tagstructbegin{tag=Document}}
\AtEndDocument{\tagstructend}
\pretocmd{\maketitle}{\tagstructbegin{tag=H1}\tagmcbegin{tag=H1}}{}{}
\apptocmd{\maketitle}{\tagmcend\tagstructend}{}{}

\begin{document}

%%%%% Frontmatter %%%%%

% Footnote symbols in front matter
\renewcommand*{\thefootnote}{\fnsymbol{footnote}}

\small
\thispagestyle{empty}
\pagenumbering{roman}
\noindent
\begin{center}
\title{Status of Vermilion rockfish (\emph{Sebastes miniatus}) along the US West - Oregon coast in 2021}
% \textnormal{\MakeTextUppercase{\trTitle{}}}
\vspace{1.5cm}
{\Large\textbf\newline{Status of Vermilion rockfish (\emph{Sebastes miniatus}) along the US West - Oregon coast in 2021}}
\vfill
by\\
Jason M. Cope\textsuperscript{1}\\
Alison D. Whitman\textsuperscript{2}\vfill
\textsuperscript{1}Northwest Fisheries Science Center, U.S. Department of Commerce, National Oceanic and Atmospheric Administration, National Marine Fisheries Service, 2725 Montlake Boulevard East, Seattle, Washington 98112\\
\textsuperscript{2}Oregon Department of Fish and Wildlife, 2040 Southeast Marine Science Drive, Newport, Oregon 97365\vfill
\trMonth{} \trYear{}
\end{center}
\clearpage

% Fourth page: Colophon
\thispagestyle{empty}
\vspace*{\fill}
\begin{center}
\copyright{} Pacific Fisheries Management Council, \trYear{}\\
\end{center}
\par
\bigskip
\noindent
Correct citation for this publication:
\bigskip
\par
\trCitation{}
\clearpage

% Add TOC to pdf bookmarks (clickable pdf)
\pdfbookmark[1]{\contentsname}{toc}

% Table of contents page, lists of figures and tables
\tableofcontents\clearpage
\label{TRlastRoman}
\clearpage

% Table of contents
\newpage
\thispagestyle{empty} % to remove page number

% Settings for the main document
\pagenumbering{arabic}  % Regular page numbers
\pagestyle{plain}  % No page number on first page of main document, use 'empty'
\renewcommand*{\thefootnote}{\arabic{footnote}}  % Back to numeric footnotes
\setcounter{footnote}{0}  % And start at 1
\renewcommand{\headrulewidth}{0.5pt}
\renewcommand{\footrulewidth}{0.5pt}
%\pagestyle{fancy}\fancyhead[c]{Draft: Do not cite or circulate}

\newcommand{\lt}{\ensuremath <}
\newcommand{\gt}{\ensuremath >}

\vspace{500cm}

\tagstructbegin{tag=H1}\tagmcbegin{tag=H1}

\hypertarget{disclaimer}{%
\section*{Disclaimer}\label{disclaimer}}
\addcontentsline{toc}{section}{Disclaimer}

\leavevmode\tagmcend\tagstructend

\tagstructbegin{tag=P}\tagmcbegin{tag=P}

\emph{\textbf{These materials do not constitute a formal publication and are for information only. They are in a pre-review, pre-decisional state and should not be formally cited or reproduced. They are to be considered provisional and do not represent any determination or policy of NOAA or the Department of Commerce.}}

\leavevmode\tagmcend\tagstructend\par

\pagebreak
\pagenumbering{roman}
\setcounter{page}{1}

\renewcommand{\thetable}{\roman{table}}
\renewcommand{\thefigure}{\roman{figure}}

\setlength\parskip{0.5em plus 0.1em minus 0.2em}

\tagstructbegin{tag=H1}\tagmcbegin{tag=H1}

\hypertarget{executive-summary}{%
\section*{Executive Summary}\label{executive-summary}}
\addcontentsline{toc}{section}{Executive Summary}

\leavevmode\tagmcend\tagstructend

\tagstructbegin{tag=H2}\tagmcbegin{tag=H2}

\hypertarget{stock}{%
\subsection*{Stock}\label{stock}}
\addcontentsline{toc}{subsection}{Stock}

\leavevmode\tagmcend\tagstructend

\tagstructbegin{tag=P}\tagmcbegin{tag=P}

This assessment reports the status of vermilion rockfish (\emph{Sebastes miniatus}) off the US West - Oregon coast using data through xxxx.

\leavevmode\tagmcend\tagstructend\par

\tagstructbegin{tag=H2}\tagmcbegin{tag=H2}

\hypertarget{landings}{%
\subsection*{Landings}\label{landings}}
\addcontentsline{toc}{subsection}{Landings}

\leavevmode\tagmcend\tagstructend

\tagstructbegin{tag=P}\tagmcbegin{tag=P}

Replace text.

\leavevmode\tagmcend\tagstructend\par

\tagstructbegin{tag=H2}\tagmcbegin{tag=H2}

\hypertarget{data-and-assessment}{%
\subsection*{Data and Assessment}\label{data-and-assessment}}
\addcontentsline{toc}{subsection}{Data and Assessment}

\leavevmode\tagmcend\tagstructend

\tagstructbegin{tag=P}\tagmcbegin{tag=P}

Replace text.

\leavevmode\tagmcend\tagstructend\par

\tagstructbegin{tag=H2}\tagmcbegin{tag=H2}

\hypertarget{stock-biomass}{%
\subsection*{Stock Biomass}\label{stock-biomass}}
\addcontentsline{toc}{subsection}{Stock Biomass}

\leavevmode\tagmcend\tagstructend

\tagstructbegin{tag=P}\tagmcbegin{tag=P}

Replace text.

\leavevmode\tagmcend\tagstructend\par

\tagstructbegin{tag=H2}\tagmcbegin{tag=H2}

\hypertarget{recruitment}{%
\subsection*{Recruitment}\label{recruitment}}
\addcontentsline{toc}{subsection}{Recruitment}

\leavevmode\tagmcend\tagstructend

\tagstructbegin{tag=P}\tagmcbegin{tag=P}

Replace text.

\leavevmode\tagmcend\tagstructend\par

\tagstructbegin{tag=H2}\tagmcbegin{tag=H2}

\hypertarget{exploitation-status}{%
\subsection*{Exploitation Status}\label{exploitation-status}}
\addcontentsline{toc}{subsection}{Exploitation Status}

\leavevmode\tagmcend\tagstructend

\tagstructbegin{tag=P}\tagmcbegin{tag=P}

Replace text.

\leavevmode\tagmcend\tagstructend\par

\tagstructbegin{tag=H2}\tagmcbegin{tag=H2}

\hypertarget{reference-points}{%
\subsection*{Reference Points}\label{reference-points}}
\addcontentsline{toc}{subsection}{Reference Points}

\leavevmode\tagmcend\tagstructend

\tagstructbegin{tag=P}\tagmcbegin{tag=P}

Replace text.

\leavevmode\tagmcend\tagstructend\par

\tagstructbegin{tag=H2}\tagmcbegin{tag=H2}

\hypertarget{management-performance}{%
\subsection*{Management Performance}\label{management-performance}}
\addcontentsline{toc}{subsection}{Management Performance}

\leavevmode\tagmcend\tagstructend

\tagstructbegin{tag=P}\tagmcbegin{tag=P}

Replace text.

\leavevmode\tagmcend\tagstructend\par

\tagstructbegin{tag=H2}\tagmcbegin{tag=H2}

\hypertarget{unresolved-problems-and-major-uncertainties}{%
\subsection*{Unresolved Problems and Major Uncertainties}\label{unresolved-problems-and-major-uncertainties}}
\addcontentsline{toc}{subsection}{Unresolved Problems and Major Uncertainties}

\leavevmode\tagmcend\tagstructend

\tagstructbegin{tag=P}\tagmcbegin{tag=P}

Replace text.

\leavevmode\tagmcend\tagstructend\par

\tagstructbegin{tag=H2}\tagmcbegin{tag=H2}

\hypertarget{decision-table}{%
\subsection*{Decision Table}\label{decision-table}}
\addcontentsline{toc}{subsection}{Decision Table}

\leavevmode\tagmcend\tagstructend

\tagstructbegin{tag=P}\tagmcbegin{tag=P}

Replace text.

\leavevmode\tagmcend\tagstructend\par

\tagstructbegin{tag=H2}\tagmcbegin{tag=H2}

\hypertarget{research-and-data-needs}{%
\subsection*{Research and Data Needs}\label{research-and-data-needs}}
\addcontentsline{toc}{subsection}{Research and Data Needs}

\leavevmode\tagmcend\tagstructend

\tagstructbegin{tag=P}\tagmcbegin{tag=P}

Replace text.

\leavevmode\tagmcend\tagstructend\par

\pagebreak
\setlength{\parskip}{5mm plus1mm minus1mm}
\pagenumbering{arabic}
\setcounter{page}{1}
\renewcommand{\thefigure}{\arabic{figure}}
\renewcommand{\thetable}{\arabic{table}}
\setcounter{table}{0}
\setcounter{figure}{0}

\setlength\parskip{0.5em plus 0.1em minus 0.2em}

\tagstructbegin{tag=H1}\tagmcbegin{tag=H1}

\hypertarget{introduction}{%
\section{Introduction}\label{introduction}}

\leavevmode\tagmcend\tagstructend

\tagstructbegin{tag=H2}\tagmcbegin{tag=H2}

\hypertarget{basic-information}{%
\subsection{Basic Information}\label{basic-information}}

\leavevmode\tagmcend\tagstructend

\tagstructbegin{tag=P}\tagmcbegin{tag=P}

This assessment reports the status of vermilion rockfish (\emph{Sebastes miniatus}) off the US West - Oregon coast using data through xxxx.

\leavevmode\tagmcend\tagstructend\par

\tagstructbegin{tag=H2}\tagmcbegin{tag=H2}

\hypertarget{life-history}{%
\subsection{Life History}\label{life-history}}

\leavevmode\tagmcend\tagstructend

\tagstructbegin{tag=P}\tagmcbegin{tag=P}

Some descriptions of vermilion rockfish were observed prior to the separation of vermilion and sunset as a cryptic species pair by Hyde et al.~in 2008 {\tagstructbegin{tag=Reference}\tagmcbegin{tag=Reference}(\textbf{Hyde2008?})\leavevmode\tagmcend\tagstructend}. Information pertaining solely to vermilion rockfish is used as much as possible.

\leavevmode\tagmcend\tagstructend\par

\tagstructbegin{tag=P}\tagmcbegin{tag=P}

Vermilion rockfish range from Prince William Sound, Alaska, to central Baja California at depths of 6 m to 436 m {\tagstructbegin{tag=Reference}\tagmcbegin{tag=Reference}(\textbf{Love2002?})\leavevmode\tagmcend\tagstructend}. However, they are most commonly found from central Oregon to Punta Baja, Mexico {\tagstructbegin{tag=Reference}\tagmcbegin{tag=Reference}(\textbf{Hyde2009?})\leavevmode\tagmcend\tagstructend} at depths of 50 m to 150 m {\tagstructbegin{tag=Reference}\tagmcbegin{tag=Reference}(\textbf{Hyde2009?})\leavevmode\tagmcend\tagstructend}. Hyde and Vetter {\tagstructbegin{tag=Reference}\tagmcbegin{tag=Reference}(\textbf{Hyde2009?})\leavevmode\tagmcend\tagstructend} describe vermilion rockfish as residents of shallower depths (\textless100 m) than sunset rockfish. Adult fish tend to cluster on high relief rocky outcrops {\tagstructbegin{tag=Reference}\tagmcbegin{tag=Reference}(\textbf{Love2002?})\leavevmode\tagmcend\tagstructend} and kelp forests {\tagstructbegin{tag=Reference}\tagmcbegin{tag=Reference}(\textbf{Hyde2009?})\leavevmode\tagmcend\tagstructend}. North of Point Conception, some adults are shallower, living in caves and cracks {\tagstructbegin{tag=Reference}\tagmcbegin{tag=Reference}(\textbf{Love2002?})\leavevmode\tagmcend\tagstructend}. Vermilion rockfish have shown high site fidelity {\tagstructbegin{tag=Reference}\tagmcbegin{tag=Reference}(\textbf{Hannah2011?} (only tagged 1 vermilion); \textbf{Lea1999?})\leavevmode\tagmcend\tagstructend}, and low average larval dispersal distance {\tagstructbegin{tag=Reference}\tagmcbegin{tag=Reference}(\textbf{Hyde2009?})\leavevmode\tagmcend\tagstructend}. Lowe et al.~(2009) {[}Lowe2009{]} suggested vermilion rockfish to have a lower site fidelity than previously believed, but they acknowledged that their observations of movements to different depths may have been due to the reality of a shallower species and a deeper species. Approximate lifespan for vermilion rockfish is 60 years, with females living longer and growing larger than their male counterparts. 50\% are mature at 5 years and about 37 cm, with males probably maturing at shorter lengths than females {\tagstructbegin{tag=Reference}\tagmcbegin{tag=Reference}(\textbf{Love2002?})\leavevmode\tagmcend\tagstructend}. Vermilion rockfish are viviparous, and release 63,000 to 2,600,000 eggs per season. In southern California, vermilion rockfish larvae are released between July and March. In central and northern California, this release occurs in September, December, and April-June {\tagstructbegin{tag=Reference}\tagmcbegin{tag=Reference}(\textbf{Love2002?})\leavevmode\tagmcend\tagstructend}. Larval release in fall and winter is not common among other rockfish species. Hyde and Vetter {\tagstructbegin{tag=Reference}\tagmcbegin{tag=Reference}(\textbf{Hyde2009?})\leavevmode\tagmcend\tagstructend} suggest that low larval dispersal may be due to weak poleward flow of nearshore waters corresponding with peak vermilion larval release. Young-of-the-year vermilion rockfish settle out of the plankton during two recruitment periods per year, first from February to April and a second from August to October, and settlement has been observed in May off southern California {\tagstructbegin{tag=Reference}\tagmcbegin{tag=Reference}(\textbf{Love2002?})\leavevmode\tagmcend\tagstructend}. Larvae measure about 4.3 mm. Both young-of-the-year vermilion and sunset rockfish are mottled brown with areas of black, and older juveniles turn a mottled orange or red color {\tagstructbegin{tag=Reference}\tagmcbegin{tag=Reference}(\textbf{Love2012?})\leavevmode\tagmcend\tagstructend}. Juvenile fish are found individually from 6 m to 36 m, living near sand and structures. After two months, juveniles travel deeper and live on low relief rocky outcrops and other structures {\tagstructbegin{tag=Reference}\tagmcbegin{tag=Reference}(\textbf{Love2002?})\leavevmode\tagmcend\tagstructend}. Adult vermilion rockfish predominantly eat smaller fish, though sometimes they pursue euphausiids and other various macroplankton {\tagstructbegin{tag=Reference}\tagmcbegin{tag=Reference}(\textbf{Phillips1964?})\leavevmode\tagmcend\tagstructend}. Love {\tagstructbegin{tag=Reference}\tagmcbegin{tag=Reference}(\textbf{Love2002?})\leavevmode\tagmcend\tagstructend} noted their diet to include octopus, salps, shrimps, and pelagic red crabs.

\leavevmode\tagmcend\tagstructend\par

\tagstructbegin{tag=H2}\tagmcbegin{tag=H2}

\hypertarget{ecosystem-considerations}{%
\subsection{Ecosystem Considerations}\label{ecosystem-considerations}}

\leavevmode\tagmcend\tagstructend

\tagstructbegin{tag=P}\tagmcbegin{tag=P}

Replace text.

\leavevmode\tagmcend\tagstructend\par

\tagstructbegin{tag=H2}\tagmcbegin{tag=H2}

\hypertarget{historical-and-current-fishery-information}{%
\subsection{Historical and Current Fishery Information}\label{historical-and-current-fishery-information}}

\leavevmode\tagmcend\tagstructend

\tagstructbegin{tag=P}\tagmcbegin{tag=P}

Replace text.

\leavevmode\tagmcend\tagstructend\par

\tagstructbegin{tag=H2}\tagmcbegin{tag=H2}

\hypertarget{summary-of-management-history-and-performance}{%
\subsection{Summary of Management History and Performance}\label{summary-of-management-history-and-performance}}

\leavevmode\tagmcend\tagstructend

\tagstructbegin{tag=P}\tagmcbegin{tag=P}

Replace text.

\leavevmode\tagmcend\tagstructend\par

\tagstructbegin{tag=H1}\tagmcbegin{tag=H1}

\hypertarget{data}{%
\section{Data}\label{data}}

\leavevmode\tagmcend\tagstructend

\tagstructbegin{tag=P}\tagmcbegin{tag=P}

A description of each data source is provided below (Figure \ref{fig:data-plot}).

\leavevmode\tagmcend\tagstructend\par

\tagstructbegin{tag=H2}\tagmcbegin{tag=H2}

\hypertarget{fishery-dependent-data}{%
\subsection{Fishery-Dependent Data}\label{fishery-dependent-data}}

\leavevmode\tagmcend\tagstructend

\tagstructbegin{tag=H2}\tagmcbegin{tag=H2}

\hypertarget{fishery-independent-data}{%
\subsection{Fishery-Independent Data}\label{fishery-independent-data}}

\leavevmode\tagmcend\tagstructend

\tagstructbegin{tag=H3}\tagmcbegin{tag=H3}

\hypertarget{section}{%
\subsubsection{\texorpdfstring{\acrlong{s-aslope}}{}}\label{section}}

\leavevmode\tagmcend\tagstructend

\tagstructbegin{tag=P}\tagmcbegin{tag=P}

The \gls{s-aslope} operated during the months of October to November aboard the R/V \emph{Miller Freeman}. Partial survey coverage of the US west coast occurred during the years 1988-1996 and complete coverage (north of 34\textdegree 30\textquotesingle S) during the years 1997 and 1999-2001. Typically, only these four years that are seen as complete surveys are included in assessments.

\leavevmode\tagmcend\tagstructend\par

\tagstructbegin{tag=H3}\tagmcbegin{tag=H3}

\hypertarget{section-1}{%
\subsubsection{\texorpdfstring{\acrlong{s-tri}}{}}\label{section-1}}

\leavevmode\tagmcend\tagstructend

\tagstructbegin{tag=P}\tagmcbegin{tag=P}

The \gls{s-tri} was first conducted by the \gls{afsc} in 1977, and the survey continued until 2004 {\tagstructbegin{tag=Reference}\tagmcbegin{tag=Reference}(Weinberg et al. 2002)\leavevmode\tagmcend\tagstructend}. Its basic design was a series of equally-spaced east-to-west transects across the continential shelf from which searches for tows in a specific depth range were initiated. The survey design changed slightly over time. In general, all of the surveys were conducted in the mid summer through early fall. The 1977 survey was conducted from early July through late September. The surveys from 1980 through 1989 were conducted from mid-July to late September. The 1992 survey was conducted from mid July through early October. The 1995 survey was conducted from early June through late August. The 1998 survey was conducted from early June through early August. Finally, the 2001 and 2004 surveys were conducted from May to July.

\leavevmode\tagmcend\tagstructend\par

\tagstructbegin{tag=P}\tagmcbegin{tag=P}

Haul depths ranged from 91-457 m during the 1977 survey with no hauls shallower than 91 m. Due to haul performance issues and truncated sampling with respect to depth, the data from 1977 were omitted from this analysis. The surveys in 1980, 1983, and 1986 covered the US West Coast south to 36.8\textdegree N latitude and a depth range of 55-366 m. The surveys in 1989 and 1992 covered the same depth range but extended the southern range to 34.5\textdegree N (near Point Conception). From 1995 through 2004, the surveys covered the depth range 55-500 m and surveyed south to 34.5\textdegree N. In 2004, the final year of the \gls{s-tri} series, the \gls{nwfsc} \gls{fram} conducted the survey following similar protocols to earlier years.

\leavevmode\tagmcend\tagstructend\par

\tagstructbegin{tag=H3}\tagmcbegin{tag=H3}

\hypertarget{section-2}{%
\subsubsection{\texorpdfstring{\acrlong{s-wcgbt}}{}}\label{section-2}}

\leavevmode\tagmcend\tagstructend

\tagstructbegin{tag=P}\tagmcbegin{tag=P}

The \Gls{s-wcgbt} is based on a random-grid design; covering the coastal waters from a depth of 55-1,280 m {\tagstructbegin{tag=Reference}\tagmcbegin{tag=Reference}(Bradburn, Keller, and Horness 2011)\leavevmode\tagmcend\tagstructend}. This design generally uses four industry-chartered vessels per year assigned to a roughly equal number of randomly selected grid cells and divided into two `passes' of the coast. Two vessels fish from north to south during each pass between late May to early October. This design therefore incorporates both vessel-to-vessel differences in catchability, as well as variance associated with selecting a relatively small number (approximately 700) of possible cells from a very large set of possible cells spread from the Mexican to the Canadian borders.

\leavevmode\tagmcend\tagstructend\par

\tagstructbegin{tag=H2}\tagmcbegin{tag=H2}

\hypertarget{biological-parameters}{%
\subsection{Biological Parameters}\label{biological-parameters}}

\leavevmode\tagmcend\tagstructend

\tagstructbegin{tag=H3}\tagmcbegin{tag=H3}

\hypertarget{growth-length-at-age}{%
\subsubsection{Growth (Length-at-Age)}\label{growth-length-at-age}}

\leavevmode\tagmcend\tagstructend

\tagstructbegin{tag=P}\tagmcbegin{tag=P}

The length-at-age was estimated for female and male vermilion rockfish using data from collections sampling the commerical and recreational fisheries off the coast of Oregon from years 2004-2020 (Table \ref{tab:len-at-age-samps}. Figure \ref{fig:len-age-data} shows the lengths and ages for all years by sex and data source as well as predicted von Bertalanffy growth function (VBGF) fits to the data. Females grow larger than males and sex-specific growth parameters were estimated at the following values:

\leavevmode\tagmcend\tagstructend\par

\begin{centering}

Females $L_{\infty}$ = 57.2 cm; $k$ = 0.146; $t_0$ = -0.65

Males $L_{\infty}$ = 54.2 cm; $k$ = 0.18; $t_0$ = 0

\end{centering}

\vspace{0.5cm}

\tagstructbegin{tag=P}\tagmcbegin{tag=P}

The estimated VBGF parameters provided initial values for the estimation of growth in the model, as all age and length data are included in the model. The resultant growth curves estimated by the model are presented in Figure \ref{fig:len-age-ss}. Sensitivity to the treatment of growth parameters (fixed or estimated) are explored through sensitivity analyses.

\leavevmode\tagmcend\tagstructend\par

\tagstructbegin{tag=H3}\tagmcbegin{tag=H3}

\hypertarget{ageing-precision-and-bias}{%
\subsubsection{Ageing Precision and Bias}\label{ageing-precision-and-bias}}

\leavevmode\tagmcend\tagstructend

\tagstructbegin{tag=P}\tagmcbegin{tag=P}

Counting ages from ageing structures in long-lived temparate fishes is challenging. Ages derived from these structures can be hard to reproduce within and between readers (i.e., imprecision), and may not contain the true age (i.e., bias). Stock assessment outputs can be affected by bias and imprecision in ageing, thus it is important to quantify and integrate this source of variability when fitting age data in assessments. In Stock Synthesis, this is done by including ageing error matrices that include the mean age (row 1) and standard deviation in age (row 2). Ageing bias is implemented When the inputted mean age deviates from the expected middle age for any given age bin (e.g., 1.75 inputted versus 1.5 being the true age); ageing imprecision is given as the standard deviation for each age bin (row 2).

\leavevmode\tagmcend\tagstructend\par

\tagstructbegin{tag=P}\tagmcbegin{tag=P}

Ageing error matrices for commerical and recreational fisheries respectively were calculated using multiple reads within each reader (n = 181 for commercial; n = 237 for recreational). An additional ageing error matrix was constructed from the Committee of Age Reading Experts (CARE) otolith exchange, where an exchange of 43 individuals was done amonth ODFW, WDFW, SWFSC, and NWFSC. The ODFW internal reads were used in the reference model, with the CARE comparison explored in a sensitivity model run.

\leavevmode\tagmcend\tagstructend\par

\tagstructbegin{tag=P}\tagmcbegin{tag=P}

Estimation of ageing error matrices for each lab used the approach of Punt et al.~(2008). The ageing error matrix offers a way to calculate both bias and imprecision in age reads. Reader 1, the primary reader of the ages used in the stock assessment, is always considered unbiased, but may be imprecise. Several model configurations are available for exploration based on either the functional form (e.g., constant CV, curvilinear standard deviation, or curvilinear CV) of the bias in reader 2 or in the precision of the readers. Model selection uses AIC corrected for small sample size (AICc), which converges to AIC when sample sizes are large. Bayesian Information Criterion (BIC) was also considered when selecting a final model.

\leavevmode\tagmcend\tagstructend\par

\tagstructbegin{tag=P}\tagmcbegin{tag=P}

The ODFW interlab comparison supported imprecision with a curvilinear standard deviation for the recretaional fishery, and a linear one for commercial. The CARE comparison was also linear, with a bit higher standard deviation (Table \textbackslash ref\{tab:age-error-models). The functional forms for each matrix are given in Figure \ref{fig:age-error}.

\leavevmode\tagmcend\tagstructend\par

\tagstructbegin{tag=H3}\tagmcbegin{tag=H3}

\hypertarget{natural-mortality}{%
\subsubsection{Natural Mortality}\label{natural-mortality}}

\leavevmode\tagmcend\tagstructend

\tagstructbegin{tag=P}\tagmcbegin{tag=P}

Natural mortality was not directly measured, so life-history based empirical relationships were used. The Natural Mortality Tool (NMT; {\tagstructbegin{tag=Link}\tagmcbegin{tag=Link}\url{https://github.com/shcaba/Natural-Mortality-Tool}\leavevmode\tagmcend\tagstructend}), a Shiny-based graphical user interface allowing for the application of a variety of natural mortality estimators based on measures such as longevity, size, age and growth, and maturity, was used to obtain estimates of natural mortality. The NMT currently provides 22 options, including the Hamel {\tagstructbegin{tag=Reference}\tagmcbegin{tag=Reference}(2015)\leavevmode\tagmcend\tagstructend} method, which is a corrected form of the Then et al. {\tagstructbegin{tag=Reference}\tagmcbegin{tag=Reference}(2015)\leavevmode\tagmcend\tagstructend} functional regression model and is a commonly applied method for west coast groundfish. The NMT also allows for the construction of a natural mortality prior weighted across methods by the user.

\leavevmode\tagmcend\tagstructend\par

\tagstructbegin{tag=P}\tagmcbegin{tag=P}

We assumed the age of 54 years to represent the practical longevity (i.e., 90\% of the commonly seen maximum age of 60) for both females and males, though the absolute oldest age in OR was \textgreater60 years. In the larger biomass, higher sampled area of California, ages 80+ were even encountered. Empirical {\tagstructbegin{tag=Formula}\tagmcbegin{tag=Formula}\(M\)\leavevmode\tagmcend\tagstructend} estimators using the von Bertalanffy growth parameters were also considered, but they produced unreasonably high estimates (2-3 times higher than the longevity estimates). This is likely explained by the fact that while vermilion rockfish have protracted longevity at {\tagstructbegin{tag=Formula}\tagmcbegin{tag=Formula}\(L_{\infty}\)\leavevmode\tagmcend\tagstructend}. Additionally, the FishLife {\tagstructbegin{tag=Reference}\tagmcbegin{tag=Reference}(\textbf{thorson\_predicting\_2017?})\leavevmode\tagmcend\tagstructend} estimate was included, though, given the source of FishLife data is FishBase, there is a good chance the estimates of {\tagstructbegin{tag=Formula}\tagmcbegin{tag=Formula}\(M\)\leavevmode\tagmcend\tagstructend} are also from methods using longevity, though the actual source of longevity in FishLife was unknown. The final composite {\tagstructbegin{tag=Formula}\tagmcbegin{tag=Formula}\(M\)\leavevmode\tagmcend\tagstructend} distributionn (Figure \ref{fig:M_composite_dists}) are based on 4 empirical estimators, and result in a median value of 0.1. We assume a lognormal distribution with a standard deviation of 0.438 ({\tagstructbegin{tag=Reference}\tagmcbegin{tag=Reference}Hamel (2015)\leavevmode\tagmcend\tagstructend}) for the purposes of the prior used to estimate {\tagstructbegin{tag=Formula}\tagmcbegin{tag=Formula}\(M\)\leavevmode\tagmcend\tagstructend}. This creates a wide prior to allow the data in the model to also influence the final estimated value of {\tagstructbegin{tag=Formula}\tagmcbegin{tag=Formula}\(M\)\leavevmode\tagmcend\tagstructend}.We also explore sensitivity to these assumptions of natural mortality through likelihood profiling.

\leavevmode\tagmcend\tagstructend\par

\tagstructbegin{tag=H3}\tagmcbegin{tag=H3}

\hypertarget{maturation-and-fecundity}{%
\subsubsection{Maturation and Fecundity}\label{maturation-and-fecundity}}

\leavevmode\tagmcend\tagstructend

\tagstructbegin{tag=P}\tagmcbegin{tag=P}

Maturity-at-length is based on the work of Hannah and Kautzi {\tagstructbegin{tag=Reference}\tagmcbegin{tag=Reference}(2012)\leavevmode\tagmcend\tagstructend} which estimated the 50 percent size-at-maturity of 39.4 cm off Oregon, though the slope of the maturity curve was not provided. Looking at the data provided in the reference, and length at 95\% maturity was assumed at 48cm, resulting in a slope of -0.34. Maturity was assumed to stay asymptotic for larger fish (Figure \ref{fig:maturity}) as no functional maturity estimate was availale {\tagstructbegin{tag=Reference}\tagmcbegin{tag=Reference}(Head, Cope, and Wulfing 2020)\leavevmode\tagmcend\tagstructend}.

\leavevmode\tagmcend\tagstructend\par

\tagstructbegin{tag=P}\tagmcbegin{tag=P}

The fecundity-at-length was based on research by Dick et al.{\tagstructbegin{tag=Reference}\tagmcbegin{tag=Reference}(2017)\leavevmode\tagmcend\tagstructend}. The fecundity relationship for vermilion rockfish was estimated equal to {\tagstructbegin{tag=Formula}\tagmcbegin{tag=Formula}\(Fec\)\leavevmode\tagmcend\tagstructend}=4.32e-07{\tagstructbegin{tag=Formula}\tagmcbegin{tag=Formula}\(L\)\leavevmode\tagmcend\tagstructend}\textsuperscript{3.55} in millions of eggs where {\tagstructbegin{tag=Formula}\tagmcbegin{tag=Formula}\(L\)\leavevmode\tagmcend\tagstructend} is length in cm. Fecundity-at-length is shown in Figure \ref{fig:fecundity}.

\leavevmode\tagmcend\tagstructend\par

\tagstructbegin{tag=H3}\tagmcbegin{tag=H3}

\hypertarget{length-weight-relationship}{%
\subsubsection{Length-Weight Relationship}\label{length-weight-relationship}}

\leavevmode\tagmcend\tagstructend

\tagstructbegin{tag=P}\tagmcbegin{tag=P}

The length(cm)-weight(kg) relationship for vermilion rockfish was estimated outside the model using all coastwide biological data available from fishery-independent data sources. The estimated length-weight relationship for female fish was {\tagstructbegin{tag=Formula}\tagmcbegin{tag=Formula}\(W\)\leavevmode\tagmcend\tagstructend}=2.60642e-05{\tagstructbegin{tag=Formula}\tagmcbegin{tag=Formula}\(L\)\leavevmode\tagmcend\tagstructend}\textsuperscript{2.93} and males at {\tagstructbegin{tag=Formula}\tagmcbegin{tag=Formula}\(W\)\leavevmode\tagmcend\tagstructend}=3.7636e-05{\tagstructbegin{tag=Formula}\tagmcbegin{tag=Formula}\(L\)\leavevmode\tagmcend\tagstructend}\textsuperscript{2.83} (Figures \ref{fig:len-weight}).

\leavevmode\tagmcend\tagstructend\par

\tagstructbegin{tag=H3}\tagmcbegin{tag=H3}

\hypertarget{sex-ratio}{%
\subsubsection{Sex Ratio}\label{sex-ratio}}

\leavevmode\tagmcend\tagstructend

\tagstructbegin{tag=P}\tagmcbegin{tag=P}

No information on the sex ratio at birth was available so it was assumed to be 50:50.

\leavevmode\tagmcend\tagstructend\par

\tagstructbegin{tag=H3}\tagmcbegin{tag=H3}

\hypertarget{steepness}{%
\subsubsection{Steepness}\label{steepness}}

\leavevmode\tagmcend\tagstructend

\tagstructbegin{tag=P}\tagmcbegin{tag=P}

The Thorson-Dorn rockfish prior (developed for use West Coast rockfish assessments) conducted by James Thorson (personal communication, NWFSC, NOAA) and reviewed and endorsed by the Scientific and Statistical Committee (SSC) in 2017, has been a primary source of information on steepness for rockfishes. This approach, however, was subsequently rejected for future analysis in 2019 when the new meta-analysis resulted in a mean value of approximately 0.95. In the absense of a new method for generating a prior for steepness the default approach reverts to the previously endorsed method, the 2017 prior for steepness ({\tagstructbegin{tag=Formula}\tagmcbegin{tag=Formula}\(h\)\leavevmode\tagmcend\tagstructend}; beta distribution with {\tagstructbegin{tag=Formula}\tagmcbegin{tag=Formula}\(\mu\)\leavevmode\tagmcend\tagstructend}=0.72 and {\tagstructbegin{tag=Formula}\tagmcbegin{tag=Formula}\(\sigma\)\leavevmode\tagmcend\tagstructend}=0.15) is retained.

\leavevmode\tagmcend\tagstructend\par

\tagstructbegin{tag=H2}\tagmcbegin{tag=H2}

\hypertarget{environmental-and-ecosystem-data}{%
\subsection{Environmental and Ecosystem Data}\label{environmental-and-ecosystem-data}}

\leavevmode\tagmcend\tagstructend

\tagstructbegin{tag=H1}\tagmcbegin{tag=H1}

\hypertarget{assessment-model}{%
\section{Assessment Model}\label{assessment-model}}

\leavevmode\tagmcend\tagstructend

\tagstructbegin{tag=H2}\tagmcbegin{tag=H2}

\hypertarget{summary-of-previous-assessments-and-reviews}{%
\subsection{Summary of Previous Assessments and Reviews}\label{summary-of-previous-assessments-and-reviews}}

\leavevmode\tagmcend\tagstructend

\tagstructbegin{tag=H3}\tagmcbegin{tag=H3}

\hypertarget{history-of-modeling-approaches-not-required-for-an-update-assessment}{%
\subsubsection{History of Modeling Approaches (not required for an update assessment)}\label{history-of-modeling-approaches-not-required-for-an-update-assessment}}

\leavevmode\tagmcend\tagstructend

\tagstructbegin{tag=H3}\tagmcbegin{tag=H3}

\hypertarget{most-recent-star-panel-and-ssc-recommendations-not-required-for-an-update-assessment}{%
\subsubsection{Most Recent STAR Panel and SSC Recommendations (not required for an update assessment)}\label{most-recent-star-panel-and-ssc-recommendations-not-required-for-an-update-assessment}}

\leavevmode\tagmcend\tagstructend

\tagstructbegin{tag=H3}\tagmcbegin{tag=H3}

\hypertarget{response-to-groundfish-subcommittee-requests-not-required-in-draft}{%
\subsubsection{Response to Groundfish Subcommittee Requests (not required in draft)}\label{response-to-groundfish-subcommittee-requests-not-required-in-draft}}

\leavevmode\tagmcend\tagstructend

\tagstructbegin{tag=H2}\tagmcbegin{tag=H2}

\hypertarget{model-structure-and-assumptions}{%
\subsection{Model Structure and Assumptions}\label{model-structure-and-assumptions}}

\leavevmode\tagmcend\tagstructend

\tagstructbegin{tag=H3}\tagmcbegin{tag=H3}

\hypertarget{model-changes-from-the-last-assessment-not-required-for-an-update-assessment}{%
\subsubsection{Model Changes from the Last Assessment (not required for an update assessment)}\label{model-changes-from-the-last-assessment-not-required-for-an-update-assessment}}

\leavevmode\tagmcend\tagstructend

\tagstructbegin{tag=H3}\tagmcbegin{tag=H3}

\hypertarget{modeling-platform-and-structure}{%
\subsubsection{Modeling Platform and Structure}\label{modeling-platform-and-structure}}

\leavevmode\tagmcend\tagstructend

\tagstructbegin{tag=P}\tagmcbegin{tag=P}

General model specifications (e.g., executable version, model structure, definition of fleets and areas)

\leavevmode\tagmcend\tagstructend\par

\tagstructbegin{tag=H3}\tagmcbegin{tag=H3}

\hypertarget{model-parameters}{%
\subsubsection{Model Parameters}\label{model-parameters}}

\leavevmode\tagmcend\tagstructend

\tagstructbegin{tag=P}\tagmcbegin{tag=P}

Describe estimated vs.~fixed parameters, priors

\leavevmode\tagmcend\tagstructend\par

\tagstructbegin{tag=H3}\tagmcbegin{tag=H3}

\hypertarget{key-assumptions-and-structural-choices}{%
\subsubsection{Key Assumptions and Structural Choices}\label{key-assumptions-and-structural-choices}}

\leavevmode\tagmcend\tagstructend

\tagstructbegin{tag=H2}\tagmcbegin{tag=H2}

\hypertarget{base-model-results}{%
\subsection{Base Model Results}\label{base-model-results}}

\leavevmode\tagmcend\tagstructend

\tagstructbegin{tag=H3}\tagmcbegin{tag=H3}

\hypertarget{parameter-estimates}{%
\subsubsection{Parameter Estimates}\label{parameter-estimates}}

\leavevmode\tagmcend\tagstructend

\tagstructbegin{tag=H3}\tagmcbegin{tag=H3}

\hypertarget{fits-to-the-data}{%
\subsubsection{Fits to the Data}\label{fits-to-the-data}}

\leavevmode\tagmcend\tagstructend

\tagstructbegin{tag=H3}\tagmcbegin{tag=H3}

\hypertarget{population-trajectory}{%
\subsubsection{Population Trajectory}\label{population-trajectory}}

\leavevmode\tagmcend\tagstructend

\tagstructbegin{tag=H3}\tagmcbegin{tag=H3}

\hypertarget{reference-points-1}{%
\subsubsection{Reference Points}\label{reference-points-1}}

\leavevmode\tagmcend\tagstructend

\tagstructbegin{tag=H2}\tagmcbegin{tag=H2}

\hypertarget{model-diagnostics}{%
\subsection{Model Diagnostics}\label{model-diagnostics}}

\leavevmode\tagmcend\tagstructend

\tagstructbegin{tag=P}\tagmcbegin{tag=P}

Describe all diagnostics

\leavevmode\tagmcend\tagstructend\par

\tagstructbegin{tag=H3}\tagmcbegin{tag=H3}

\hypertarget{convergence}{%
\subsubsection{Convergence}\label{convergence}}

\leavevmode\tagmcend\tagstructend

\tagstructbegin{tag=H3}\tagmcbegin{tag=H3}

\hypertarget{sensitivity-analyses}{%
\subsubsection{Sensitivity Analyses}\label{sensitivity-analyses}}

\leavevmode\tagmcend\tagstructend

\tagstructbegin{tag=H3}\tagmcbegin{tag=H3}

\hypertarget{retrospective-analysis}{%
\subsubsection{Retrospective Analysis}\label{retrospective-analysis}}

\leavevmode\tagmcend\tagstructend

\tagstructbegin{tag=H3}\tagmcbegin{tag=H3}

\hypertarget{likelihood-profiles}{%
\subsubsection{Likelihood Profiles}\label{likelihood-profiles}}

\leavevmode\tagmcend\tagstructend

\tagstructbegin{tag=H3}\tagmcbegin{tag=H3}

\hypertarget{unresolved-problems-and-major-uncertainties-1}{%
\subsubsection{Unresolved Problems and Major Uncertainties}\label{unresolved-problems-and-major-uncertainties-1}}

\leavevmode\tagmcend\tagstructend

\tagstructbegin{tag=H1}\tagmcbegin{tag=H1}

\hypertarget{management}{%
\section{Management}\label{management}}

\leavevmode\tagmcend\tagstructend

\tagstructbegin{tag=H2}\tagmcbegin{tag=H2}

\hypertarget{reference-points-2}{%
\subsection{Reference Points}\label{reference-points-2}}

\leavevmode\tagmcend\tagstructend

\tagstructbegin{tag=H2}\tagmcbegin{tag=H2}

\hypertarget{unresolved-problems-and-major-uncertainties-2}{%
\subsection{Unresolved Problems and Major Uncertainties}\label{unresolved-problems-and-major-uncertainties-2}}

\leavevmode\tagmcend\tagstructend

\tagstructbegin{tag=H2}\tagmcbegin{tag=H2}

\hypertarget{harvest-projections-and-decision-tables}{%
\subsection{Harvest Projections and Decision Tables}\label{harvest-projections-and-decision-tables}}

\leavevmode\tagmcend\tagstructend

\tagstructbegin{tag=H2}\tagmcbegin{tag=H2}

\hypertarget{evaluation-of-scientific-uncertainty}{%
\subsection{Evaluation of Scientific Uncertainty}\label{evaluation-of-scientific-uncertainty}}

\leavevmode\tagmcend\tagstructend

\tagstructbegin{tag=H2}\tagmcbegin{tag=H2}

\hypertarget{research-and-data-needs-1}{%
\subsection{Research and Data Needs}\label{research-and-data-needs-1}}

\leavevmode\tagmcend\tagstructend

\tagstructbegin{tag=H1}\tagmcbegin{tag=H1}

\hypertarget{acknowledgments}{%
\section{Acknowledgments}\label{acknowledgments}}

\leavevmode\tagmcend\tagstructend

\tagstructbegin{tag=P}\tagmcbegin{tag=P}

Here are all the mad props!

\leavevmode\tagmcend\tagstructend\par

\clearpage

\tagstructbegin{tag=H1}\tagmcbegin{tag=H1}

\hypertarget{references}{%
\section{References}\label{references}}

\leavevmode\tagmcend\tagstructend

\tagstructbegin{tag=BibEntry}\tagmcbegin{tag=BibEntry}

\hypertarget{refs}{}
\begin{CSLReferences}{1}{0}
\leavevmode\hypertarget{ref-bradburn_2003_2011}{}%
Bradburn, M. J., A. A Keller, and B. H. Horness. 2011. {``The 2003 to 2008 {US} {West} {Coast} Bottom Trawl Surveys of Groundfish Resources Off {Washington}, {Oregon}, and {California}: Estimates of Distribution, Abundance, Length, and Age Composition.''} US Department of Commerce, National Oceanic; Atmospheric Administration, National Marine Fisheries Service.

\leavevmode\hypertarget{ref-dick_meta-analysis_2017}{}%
Dick, E. J., Sabrina Beyer, Marc Mangel, and Stephen Ralston. 2017. {``A Meta-Analysis of Fecundity in Rockfishes (Genus \emph{Sebastes}).''} \emph{Fisheries Research} 187 (March): 73--85. \url{https://doi.org/10.1016/j.fishres.2016.11.009}.

\leavevmode\hypertarget{ref-hamel_method_2015}{}%
Hamel, Owen S. 2015. {``A Method for Calculating a Meta-Analytical Prior for the Natural Mortality Rate Using Multiple Life History Correlates.''} \emph{ICES Journal of Marine Science: Journal Du Conseil} 72 (1): 62--69. \url{https://doi.org/10.1093/icesjms/fsu131}.

\leavevmode\hypertarget{ref-hannah_age_2012}{}%
Hannah, Robert W, and Lisa A Kautzi. 2012. {``Age, Growth and Female Maturity of Vermilion Rockfish (Sebastes Miniatus) from Oregon Waters,''} 20.

\leavevmode\hypertarget{ref-head_fxnalmatspline_2020}{}%
Head, Melissa A., Jason M. Cope, and Sophie H. Wulfing. 2020. {``Applying a Flexible Spline Model to Estimate Functional Maturity and Spatio-Temporal Variability in Aurora Rockfish (Sebastes Aurora).''} \emph{Environmental Biology of Fishes}. \url{https://doi.org/10.1007/s10641-020-01014-2}.

\leavevmode\hypertarget{ref-then_evaluating_2015-1}{}%
Then, A. Y., J. M. Hoenig, N. G. Hall, and D. A. Hewitt. 2015. {``Evaluating the Predictive Performance of Empirical Estimators of Natural Mortality Rate Using Information on over 200 Fish Species.''} \emph{ICES Journal of Marine Science} 72 (1): 82--92. \url{https://doi.org/10.1093/icesjms/fsu136}.

\leavevmode\hypertarget{ref-weinberg_2001_2002}{}%
Weinberg, K. L., M. E. Wilkins, F. R. Shaw, and M. Zimmermann. 2002. {``The 2001 {Pacific} {West} {Coast} Bottom Trawl Survey of Groundfish Resources: Estimates of Distribution, Abundance and Length and Age Composition.''} NOAA Technical Memorandum NMFS-AFSC-128. U.S. Department of Commerce.

\end{CSLReferences}

\leavevmode\tagmcend\tagstructend

\clearpage

\tagstructbegin{tag=H1}\tagmcbegin{tag=H1}

\hypertarget{tables}{%
\section{Tables}\label{tables}}

\leavevmode\tagmcend\tagstructend

\begingroup\fontsize{10}{12}\selectfont
\begingroup\fontsize{10}{12}\selectfont

\begin{longtable}[t]{r>{\centering\arraybackslash}p{2cm}>{\centering\arraybackslash}p{2cm}}
\caption{\label{tab:allcatches}Catches (mt) by fleet for all years and total catches (mt) summed by year.}\\
\toprule
Year & Fishery & Total Catch\\
\midrule
\endfirsthead
\caption[]{Catches (mt) by fleet for all years and total catches (mt) summed by year. \textit{(continued)}}\\
\toprule
Year & Fishery & Total Catch\\
\midrule
\endhead

\endfoot
\bottomrule
\endlastfoot
1949 & 0.00 & 0.00\\
1950 & 0.01 & 0.01\\
1951 & 0.01 & 0.01\\
1952 & 0.03 & 0.03\\
1953 & 0.02 & 0.02\\
1954 & 0.04 & 0.04\\
1955 & 0.04 & 0.04\\
1956 & 0.07 & 0.07\\
1957 & 0.08 & 0.08\\
1958 & 0.06 & 0.06\\
1959 & 0.09 & 0.09\\
1960 & 0.04 & 0.04\\
1961 & 0.14 & 0.14\\
1962 & 0.12 & 0.12\\
1963 & 0.10 & 0.10\\
1964 & 0.08 & 0.08\\
1965 & 0.17 & 0.17\\
1966 & 0.14 & 0.14\\
1967 & 1.40 & 1.40\\
1968 & 0.14 & 0.14\\
1969 & 0.14 & 0.14\\
1970 & 0.14 & 0.14\\
1971 & 0.15 & 0.15\\
1972 & 0.15 & 0.15\\
1973 & 0.15 & 0.15\\
1974 & 0.15 & 0.15\\
1975 & 0.15 & 0.15\\
1976 & 0.09 & 0.09\\
1977 & 0.22 & 0.22\\
1978 & 0.19 & 0.19\\
1979 & 0.14 & 0.14\\
1980 & 0.17 & 0.17\\
1981 & 0.20 & 0.20\\
1982 & 0.10 & 0.10\\
1983 & 0.33 & 0.33\\
1984 & 0.54 & 0.54\\
1985 & 0.43 & 0.43\\
1986 & 0.28 & 0.28\\
1987 & 0.44 & 0.44\\
1988 & 0.54 & 0.54\\
1989 & 1.09 & 1.09\\
1990 & 0.91 & 0.91\\
1991 & 1.45 & 1.45\\
1992 & 1.48 & 1.48\\
1993 & 1.45 & 1.45\\
1994 & 1.06 & 1.06\\
1995 & 0.87 & 0.87\\
1996 & 0.84 & 0.84\\
1997 & 0.71 & 0.71\\
1998 & 0.49 & 0.49\\
1999 & 0.79 & 0.79\\
2000 & 0.64 & 0.64\\
2001 & 0.56 & 0.56\\
2002 & 0.30 & 0.30\\
2003 & 0.22 & 0.22\\
2004 & 0.19 & 0.19\\
2005 & 0.38 & 0.38\\
2006 & 0.44 & 0.44\\
2007 & 0.95 & 0.95\\
2008 & 0.56 & 0.56\\
2009 & 0.36 & 0.36\\
2010 & 0.74 & 0.74\\
2011 & 1.01 & 1.01\\
2012 & 0.95 & 0.95\\
2013 & 1.05 & 1.05\\
2014 & 1.04 & 1.04\\
2015 & 1.32 & 1.32\\
2016 & 0.82 & 0.82\\
2017 & 0.97 & 0.97\\
2018 & 1.24 & 1.24\\
2019 & 2.60 & 2.60\\
2020 & 0.66 & 0.66\\*
\end{longtable}
\endgroup{}
\endgroup{}


\newpage

\begingroup\fontsize{9}{11}\selectfont

\begin{landscape}\begingroup\fontsize{9}{11}\selectfont

\begin{longtable}[t]{c>{\centering\arraybackslash}p{1cm}>{\centering\arraybackslash}p{1cm}>{\centering\arraybackslash}p{0.1cm}>{\centering\arraybackslash}p{1cm}>{\centering\arraybackslash}p{0.1cm}>{\centering\arraybackslash}p{1cm}>{\centering\arraybackslash}p{1cm}>{\centering\arraybackslash}p{1cm}>{\centering\arraybackslash}p{1cm}>{\centering\arraybackslash}p{1cm}}
\caption{\label{tab:ageing_error_mods}Ageing error models and resultant model selection (AICc) values for 9 models of bias and precision explored for each lab used in the vermilion rockfish assessments. Gray bars indicate the chosen model. Model codes: 0= unbiased; 1 = Constant CV; 2 = Curvilinear SD; 3= Curvilinear CV}\\
\toprule
 &  &  &  &  &  &  &  &  &  & \\
\midrule
\endfirsthead
\caption[]{Ageing error models and resultant model selection (AICc) values for 9 models of bias and precision explored for each lab used in the vermilion rockfish assessments. Gray bars indicate the chosen model. Model codes: 0= unbiased; 1 = Constant CV; 2 = Curvilinear SD; 3= Curvilinear CV \textit{(continued)}}\\
\toprule
 &  &  &  &  &  &  &  &  &  & \\
\midrule
\endhead

\endfoot
\bottomrule
\endlastfoot
Model & Bias & Precision & & Bias & Precision & & AICc & $\Delta$AICc & BIC & $\Delta$BIC\\
1 & 0 & 1 &  & 0 & 1 &  & 0 & 26 & 0 & 25\\
2 & 0 & 2 &  & 0 & 2 &  & 0 & 4 & 0 & 4\\
\textbf3 & \textbf0 & \textbf3 &  & \textbf0 & \textbf3 &  & \textbf0 & \textbf0 & \textbf0 & \textbf0\\
4 & 0 & 1 &  & 1 & 1 &  & 0 & 16 & 0 & 16\\
5 & 0 & 2 &  & 1 & 2 &  & 0 & 15 & 0 & 16\\
6 & 0 & 3 &  & 1 & 3 &  & 0 & 15 & 0 & 16\\
7 & 0 & 1 &  & 2 & 1 &  & 0 & 24 & 0 & 25\\
8 & 0 & 2 &  & 2 & 2 &  & 0 & 24 & 0 & 26\\
9 & 0 & 3 &  & 2 & 3 &  & 0 & 28 & 0 & 30\\*
\end{longtable}

\begin{longtable}[t]{c>{\centering\arraybackslash}p{1cm}>{\centering\arraybackslash}p{1cm}>{\centering\arraybackslash}p{0.1cm}>{\centering\arraybackslash}p{1cm}>{\centering\arraybackslash}p{0.1cm}>{\centering\arraybackslash}p{1cm}>{\centering\arraybackslash}p{1cm}>{\centering\arraybackslash}p{1cm}>{\centering\arraybackslash}p{1cm}>{\centering\arraybackslash}p{1cm}}
\toprule
 &  &  &  &  &  &  &  &  &  & \\
\midrule
\endfirsthead
\toprule
 &  &  &  &  &  &  &  &  &  & \\
\midrule
\endhead

\endfoot
\bottomrule
\endlastfoot
Model & Bias & Precision & & Bias & Precision & & AICc & $\Delta$AICc & BIC & $\Delta$BIC\\
\textbf1 & \textbf0 & \textbf1 &  & \textbf0 & \textbf1 &  & \textbf0 & \textbf0 & \textbf0 & \textbf0\\
2 & 0 & 2 &  & 0 & 2 &  & 0 & 4 & 0 & 6\\
3 & 0 & 3 &  & 0 & 3 &  & 0 & 4 & 0 & 6\\
4 & 0 & 1 &  & 1 & 1 &  & 0 & 0 & 0 & 3\\
5 & 0 & 2 &  & 1 & 2 &  & 0 & 4 & 0 & 8\\
6 & 0 & 3 &  & 1 & 3 &  & 0 & 8 & 0 & 12\\
7 & 0 & 1 &  & 2 & 1 &  & 0 & 39 & 0 & 42\\
8 & 0 & 2 &  & 2 & 2 &  & 0 & 10 & 0 & 14\\
9 & 0 & 3 &  & 2 & 3 &  & 0 & 9 & 0 & 14\\*
\end{longtable}

\begin{longtable}[t]{c>{\centering\arraybackslash}p{1cm}>{\centering\arraybackslash}p{1cm}>{\centering\arraybackslash}p{0.1cm}>{\centering\arraybackslash}p{1cm}>{\centering\arraybackslash}p{0.1cm}>{\centering\arraybackslash}p{1cm}>{\centering\arraybackslash}p{0.25cm}>{\centering\arraybackslash}p{1cm}>{\centering\arraybackslash}p{1cm}>{\centering\arraybackslash}p{1cm}}
\caption[]{Ageing error models and resultant model selection (AICc) values for 9 models of bias and precision explored for each lab used in the vermilion rockfish assessments. Gray bars indicate the chosen model. Model codes: 0= unbiased; 1 = Constant CV; 2 = Curvilinear SD; 3= Curvilinear CV \textit{(continued)}}\\
\toprule
 &  &  &  &  &  &  &  &  &  & \\
\midrule
\endfirsthead
\toprule
 &  &  &  &  &  &  &  &  &  & \\
\midrule
\endhead

\endfoot
\bottomrule
\endlastfoot
Model & Bias & Precision & & Bias & Precision & & AICc & $\Delta$AICc & BIC & $\Delta$BIC\\
1 & 0 & 1 &  & 0 & 1 &  & 0 & 73 & 0 & 64\\
2 & 0 & 2 &  & 0 & 2 &  & 0 & 61 & 0 & 54\\
3 & 0 & 3 &  & 0 & 3 &  & 0 & 57 & 0 & 50\\
\textbf4 & \textbf0 & \textbf1 &  & \textbf1 & \textbf1 &  & \textbf0 & \textbf0 & \textbf0 & \textbf0\\
5 & 0 & 2 &  & 1 & 2 &  & 0 & 17 & 0 & 18\\
6 & 0 & 3 &  & 1 & 3 &  & 0 & 7 & 0 & 8\\
7 & 0 & 1 &  & 2 & 1 &  & 0 & 1 & 0 & 3\\
8 & 0 & 2 &  & 2 & 2 &  & 0 & 13 & 0 & 16\\
9 & 0 & 3 &  & 2 & 3 &  & 0 & 10 & 0 & 13\\*
\end{longtable}
\endgroup{}
\end{landscape}
\endgroup{}


\newpage

\clearpage

\tagstructbegin{tag=H1}\tagmcbegin{tag=H1}

\hypertarget{figures}{%
\section{Figures}\label{figures}}

\leavevmode\tagmcend\tagstructend

\tagstructbegin{tag=Figure,alttext={Summary of data sources used in the base model.}}\tagmcbegin{tag=Figure}

\begin{figure}
\centering
\includegraphics[width=1\textwidth,height=1\textheight]{//nwcfile/FRAM/Assessments/CurrentAssessments/vermilion_2021/OR/models/Reference model/plots/data_plot.png}
\caption{Summary of data sources used in the base model.\label{fig:data-plot}}
\end{figure}

\tagmcend\tagstructend

\tagstructbegin{tag=Figure,alttext={Observed length-at-age by data source and sex. Lines indicate fits to the von Bertalanffy growth equation, with parameter estimates provided in the bottom right corner of the figure.}}\tagmcbegin{tag=Figure}

\begin{figure}
\centering
\includegraphics[width=1\textwidth,height=1\textheight]{//nwcfile/FRAM/Assessments/CurrentAssessments/vermilion_2021/OR/biology_plots/AG_plot_lines_parameters.png}
\caption{Observed length-at-age by data source and sex. Lines indicate fits to the von Bertalanffy growth equation, with parameter estimates provided in the bottom right corner of the figure.\label{fig:len-age-data}}
\end{figure}

\tagmcend\tagstructend

\tagstructbegin{tag=Figure,alttext={Length at age in the beginning of the year in the ending year of the model.}}\tagmcbegin{tag=Figure}

\begin{figure}
\centering
\includegraphics[width=1\textwidth,height=1\textheight]{//nwcfile/FRAM/Assessments/CurrentAssessments/vermilion_2021/OR/models/Reference model/plots/bio1_sizeatage.png}
\caption{Length at age in the beginning of the year in the ending year of the model.\label{fig:len-age-ss}}
\end{figure}

\tagmcend\tagstructend

\tagstructbegin{tag=Figure,alttext={Agein error matrix (age by standard deviation) values by source. The commercial and recreational matrices are based on inter-reader comparisons.}}\tagmcbegin{tag=Figure}

\begin{figure}
\centering
\includegraphics[width=1\textwidth,height=1\textheight]{//nwcfile/FRAM/Assessments/CurrentAssessments/vermilion_2021/OR/biology_plots/Age_error_plot.png}
\caption{Agein error matrix (age by standard deviation) values by source. The commercial and recreational matrices are based on inter-reader comparisons.\label{fig:age-error}}
\end{figure}

\tagmcend\tagstructend

\tagstructbegin{tag=Figure,alttext={Maturity as a function of  length.}}\tagmcbegin{tag=Figure}

\begin{figure}
\centering
\includegraphics[width=1\textwidth,height=1\textheight]{//nwcfile/FRAM/Assessments/CurrentAssessments/vermilion_2021/OR/models/Reference model/plots/bio6_maturity.png}
\caption{Maturity as a function of length.\label{fig:maturity}}
\end{figure}

\tagmcend\tagstructend

\tagstructbegin{tag=Figure,alttext={Fecundity as a function of length.}}\tagmcbegin{tag=Figure}

\begin{figure}
\centering
\includegraphics[width=1\textwidth,height=1\textheight]{//nwcfile/FRAM/Assessments/CurrentAssessments/vermilion_2021/OR/models/Reference model/plots/bio9_fecundity_len.png}
\caption{Fecundity as a function of length.\label{fig:fecundity}}
\end{figure}

\tagmcend\tagstructend

\tagstructbegin{tag=Figure,alttext={Composite natural mortality distriubtion for $S. hopkinsi$ using four longevity estimators each with a SD = 0.2 presuming a lognomral error distibution.}}\tagmcbegin{tag=Figure}

\begin{figure}
\centering
\includegraphics[width=1\textwidth,height=1\textheight]{//nwcfile/FRAM/Assessments/CurrentAssessments/vermilion_2021/OR/biology_plots/Mdensityplots_OR_vermilion.png}
\caption{Composite natural mortality distriubtion for {\tagstructbegin{tag=Formula}\tagmcbegin{tag=Formula}\(S. hopkinsi\)\leavevmode\tagmcend\tagstructend} using four longevity estimators each with a SD = 0.2 presuming a lognomral error distibution.\label{fig:M_composite_dists}}
\end{figure}

\tagmcend\tagstructend
\end{document}
